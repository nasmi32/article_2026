\begin{Russian}
\section{Обсуждение}
\label{sec:discussion}

Эксперименты показали, что Containerlab эффективнее использует хост и имеет более высокую сетевую активность. Это позволяет задавать большое количество узлов и горизонтально масштабировать сеть. В инструменте хорошо исследуются топологии, где сделан акцент на сетевой обмен, в том числе, и задачи кибербезопасности.

Преимущество Kathará заключается в стабильности нагрузки контейнеров. Следовательно, эмулятор полезен для изолированных вычислений или систем с низким сетевым трафиком. Если цель исследования состоит в мониторинге трендов, то из-за более длительного сбора данных Kathará также может быть выбрана для решения задачи.

\end{Russian}

\begin{English}
\section{Discussion}
\label{sec:discussion}

Experiments have shown that Containerlab uses the host more efficiently and has higher network activity. This allows you to set a large number of nodes and scale the network horizontally. The tool is good for exploring topologies where the emphasis is on network exchange, including cybersecurity tasks.

The advantage of Kathará is the stability of container loads. Consequently, the emulator is useful for isolated computing or systems with low network traffic. If the goal of the research is to monitor trends, Kathará may also be chosen for the task due to its longer data collection time.

\end{English}


%%% Local Variables:
%%% mode: latex
%%% coding: utf-8-unix
%%% End:
