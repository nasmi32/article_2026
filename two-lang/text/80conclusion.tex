\begin{Russian}
\section{Заключение}
\end{Russian}
\begin{English}
\section{Conclusion}
\end{English}
\label{sec:conclusion}

\begin{Russian}
Исследование было посвящено моделированию эпидемиалогических моделей в эмуляторах Kathara и Containerlab c целью их дальнейшего сравнения и анализа параметров систем. Получилось создать воспроизводимые эксперименты в двух средах, результаты которых оказались сопоставимы и похожи. Поведение теоретических моделей не выдало непредсказуемых результатов. Для исследуемых процессов технически сильнее оказался эмулятор ,,,,,,,ВСТАВИТЬ,,,,,,, поэтому в дальнейшем он будет приоритетным для решения задач кибербезопасности.
\end{Russian}
\begin{English}
The study focused on modeling epidemiological models in Kathara and Containerlab emulators for the purpose of further comparison and analysis of system parameters. We were able to create reproducible experiments in two environments, the results of which were comparable and similar. The behavior of theoretical models did not produce unpredictable results. For the processes under study, the emulator ,,,,,,,INSERT,,,,,,, proved to be technically stronger, so it will be prioritized for cybersecurity tasks in the future.
\end{English}


%%% Local Variables:
%%% mode: latex
%%% coding: utf-8-unix
%%% End:
