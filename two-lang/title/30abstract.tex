\begin{Russian}
\begin{abstract}

% \begin{description}
% \textsc{Предпосылки.} 
%
%\textsc{Цель.} 
%
%\textsc{Методы.} 
%
%\textsc{Результаты.} 
%
%\textsc{Выводы.} 
% \end{description}

  \textbf{Предпосылки} Эпидемиологические модели SIR, SEIR и SEIRS широко используются в исследовательских работах. Применение теоретических моделей заражения в работе позволяет сконцентрироваться на изучении системных параметров эмулятора Kathara, а не на математической точности результатов эмуляции. Изучение средства моделирования необходимо для его использования в будущих исследованиях. \textbf{Цель} Проанализировать технические показатели производительности средства натурного моделирования Kathara на примере распространения компьютерного вируса. \textbf{Методы} Компартментные эпидемиологические моделей SIR, SEIR и SEIRS используются для анализа динамики распространения инфекций и оценки параметров системы в условиях воспроизводимого натурного эксперимента в Kathara. Построены дискретно-временные реализации моделей с параметрами $\beta$, $\sigma$ и $\gamma$. Вычислительный эксперимент организован как сеть взаимодействующих узлов, один из которых является нулевым пациентом и запускает процесс заражения. \textbf{Результаты} Были построены и проанализированы графики трех эпидемиологических моделей. Системные метрики показали более быструю динамику симуляции на средстве моделирования Conteinerlab. \textbf{Заключение} Для исследуемых процессов Containerlab оказался технически сильнее.
 
\end{abstract}
\end{Russian}

\begin{English}
\begin{abstract}
  \textbf{Background} SIR, SEIR, and SEIRS epidemiological models are widely used in research. The use of theoretical infection models in this work allows us to focus on studying the system parameters of the Kathara emulator rather than on the mathematical accuracy of the emulation results. Studying the simulation tool is necessary for its use in future research. \textbf{Purpose} To analyze the technical performance indicators of the Kathara real-world simulation tool using the example of a computer virus outbreak. \textbf{Methods} Compartmental epidemiological models SIR, SEIR, and SEIRS are used to analyze the dynamics of infection spread and evaluate system parameters in a reproducible real-world experiment in Kathara. Discrete-time implementations of the models with parameters $\beta$, $\sigma$, and $\gamma$ are constructed. The computational experiment is organized as a network of interacting nodes, one of which is the index patient and initiates the infection process. \textbf{Results} Graphs of three epidemiological models were constructed and analyzed. System metrics showed faster simulation dynamics on the Conteinerlab simulation tool. \textbf{Conclusion} Containerlab proved to be technically superior for the processes under study.
\end{abstract}
\end{English}




